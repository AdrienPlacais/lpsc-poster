% Based on Gemini theme
% https://github.com/anishathalye/gemini
\documentclass[final]{beamer}

% ====================
% Packages (default)
% ====================
\usepackage[T1]{fontenc}
\usepackage{lmodern}
\usepackage[orientation=portrait, size=a0]{beamerposter}
\usetheme{lpscposter}
\usecolortheme{lpscposter}
\usepackage{graphicx}
\usepackage{booktabs}
% \usepackage{tabularx}

% ====================
% Packages (added)
% ====================
\usepackage{lipsum}
\usepackage{siunitx}
\DeclareSIUnit\MVm{\mega\volt\per\metre}

\usepackage{todonotes}
\presetkeys%
    {todonotes}%
    {inline}{}
 \usepackage{xspace}
 \usepackage{subfig}
 \usepackage{nth}
 \usepackage[style=ieee]{biblatex}
\renewcommand*{\bibfont}{\small}
% ====================
% Lengths
% ====================
% If you have N columns, choose \sepwidth and \colwidth such that
% (N+1)*\sepwidth + N*\colwidth = \paperwidth
\newlength{\sepwidth}
\newlength{\colwidth}

\setlength{\sepwidth}{0.025\paperwidth}
\setlength{\colwidth}{0.4625\paperwidth}

% Default for three columns:
% \setlength{\sepwidth}{0.025\paperwidth}
% \setlength{\colwidth}{0.3\paperwidth}

\newcommand{\separatorcolumn}{\begin{column}{\sepwidth}\end{column}}

% Increase column separator in tabular
\setlength{\tabcolsep}{12pt}

% ====================
% Shortcuts
% ====================
\newcommand\cf{\emph{cf}\xspace}
\newcommand\eg{e.g.\xspace}
\newcommand\etou{and\slash or\xspace}
\newcommand\etal{\emph{et al.}\xspace}
\newcommand\etc{\emph{etc}\xspace}
\newcommand\ie{\emph{i.e.}\xspace}
\newcommand\noref{[\textbf{Ref ?}]}

\newcommand\reffig[1]{Fig.~\ref{fig:#1}\xspace}
\newcommand\reftab[1]{Tab.~\ref{tab:#1}\xspace}

% Phys
\newcommand\acc{\mathrm{acc}}
\newcommand\Eacc{E_\acc}

% ====================
% Title
% ====================
\title{My poster}

\author{A. Plaçais \inst{1} \and F. Bouly \inst{1}}
\institute[shortinst]{%
   \inst{1} Univ. Grenoble Alpes, CNRS, Grenoble INP*, LPSC--IN2P3, 38000 Grenoble, France
}

% ====================
% Footer (optional)
% ====================
\footercontent{
   *Institute of Engineering Univ. Grenoble Alpes \hfill
   IPAC2023, Venice \hfill
   \href{mailto:placais@lpsc.in2p3.fr}{placais@lpsc.in2p3.fr}
}
% (can be left out to remove footer)

% ====================
% Logo (optional)
% ====================
\newlength{\logoheight}
\setlength{\logoheight}{6cm}

\newlength{\logoswidth}
\setlength{\logoswidth}{13cm}

\logoright{%
   \includegraphics[height=\logoheight]{images/LOGO_CNRS_2019_RVB.png}
   % \begin{minipage}[h]{\logoswidth}
      % \centering
      % \includegraphics[height=\logoheight]{images/jaea.png}
      % \vskip\floatsep
      % \includegraphics[height=\logoheight]{images/logo_ipac23.png}
   % \end{minipage}
}
\logoleft{%
   \includegraphics[height=\logoheight]{images/LPSC_Quadri.png}
   % \begin{minipage}[h]{\logoswidth}
      % \centering
      % \includegraphics[height=\logoheight]{images/LPSC_Quadri.png}
      % \vskip\floatsep
      % \includegraphics[height=\logoheight]{images/LOGO_CNRS_2019_RVB.png}
   % \end{minipage}
}

% -------------------------------------------------
% General
% -------------------------------------------------
% Normal figures
\newcommand\fheight{.25*\colwidth}
\newcommand\fwidth{\colwidth-\sepwidth}

% -------------------------------------------------
% Tikz
% -------------------------------------------------
\usepackage{tikz}                      % Generate plot from csv
\usepackage{tikzscale}
% \usetikzlibrary{positioning}           % Better handling of tikz pictures in subfigures
% \usetikzlibrary{spy}                   % Local zooms
\usetikzlibrary{arrows.meta}           % New arrow heads
% \usetikzlibrary{shapes}
% \usetikzlibrary{calc} % To calculate position between two nodes
% \usetikzlibrary{shapes.geometric}
% \usetikzlibrary{fit} % To compute rectangle encompassing several nodes

% -------------------------------------------------
% Pgf plots, to generate plots from data files
% -------------------------------------------------
\usepackage{pgfplots}
\usetikzlibrary{pgfplots.colorbrewer}  % For ColorBrewer, a sweet colors generator
\usetikzlibrary{pgfplots.groupplots}   % For groups of plots (equivalent of subfig for pgf plots)
\usepackage{pgfplotstable}
\usepgfplotslibrary{units}             % For sweet units in pgfplots

% Set personnalized color and marker list
\pgfplotsset{
   compat=newest,
   %---------------------------------------
   % Define style for classic plots
   %---------------------------------------
   enlarge x limits=false, 	% Prevent margins between first/last data and left/right axis.
   x tick label style={/pgf/number format/.cd,set thousands separator={\,}}, % To avoid 1 000 written as 1,000
   y tick label style={/pgf/number format/.cd,set thousands separator={\,}}, % To avoid 1 000 written as 1,000
   legend style={fill opacity=0.8, draw opacity=1, text opacity=1, draw=white!80.0!black},
   cycle list/Dark2-8, 			% Set the list of colors
   cycle multiindex* list={%	            % In order to also have different markers
      mark=*,mark=*,mark=*,mark=*\nextlist% List of markers
      Dark2-8\nextlist 		               % List of colors
      },
   normalplot/.style={%
      width=\fwidth, height=\fheight,
      legend style={fill opacity=0.8, draw opacity=1, text opacity=1, draw=white!80.0!black},
      legend cell align={left},
      legend pos=north west,
      grid=major,
      mark size=3pt,
      every axis plot/.append style={ultra thick},
   },
}

\tikzset{%
   myhead/.tip = {Stealth[scale=2]},
   arrow/.style = {-{Stealth[scale=1.5]}, thick},
   basetext/.style = {rectangle, anchor=south, font=\normalsize, fill=white, midway, below=1cm},
   blockcomment/.style = {basetext, minimum height=2em, minimum width=2em,
                          fill opacity=0.8, draw opacity=1, text opacity=1, draw=white!80.0!LPSCdarkblue, fill=LPSCbg,
                          above=1.0cm, anchor=south, font=\small,
                         },
}

% -------------------------------------------------
% Externalization
% -------------------------------------------------
\usepgfplotslibrary{external}     % Auto save tikz plots to avoid recompilation each time
\tikzexternalize[
   only named=true,     % Activate externalization, with auto-update of changed .tikz files
   % mode=list and make,  % Skips figure that needs a recompilation, but produce a makefile to create figures separately.
                        % make -f main.makefile -j 4 will produce figures with 4 parallel cores.
   mode=convert with system call,
]
% To save externalized in a separated folder:
\tikzsetexternalprefix{figures/external/}

% \addbibresource{main.bib}
\begin{document}

\begin{frame}[t]
   \begin{columns}[t]
      \separatorcolumn
      \begin{column}{\colwidth}
         \begin{block}{Why do we want to compensate cavity failures?}
            The RF cavities in LINACs are subject to failures, which can lead to beam losses.
            It is a major problem for Accelerator Driven Systems (ADS), as every beam trip cause adverse thermal stress on the structure and as the restart procedure can be time-consuming.
            A faulty cavity can be compensated for by retuning it's neighboring cavities.
            \begin{figure}
               \centering
               \includegraphics[width=\columnwidth,height=.2\columnwidth]{example-image-a}
               \caption{%
                  Local compensation of a failed cavity (red).
                  The \num{5} neighboring cavities (orange) compensate for the fault; cavities in green remain untouched.
               }
               \label{fig:compensation_scheme}
            \end{figure}
            Several tools such as TraceWin can find compensation settings.
            However, the process require a lot of manual work, which is time-consuming and error-prone.

            Hence we developed LightWin, a tool to automatically find compensation settings, already tested on the MYRRHA--ADS \alert{(i) does it work for another linac?}

            It is a longitudinal beam dynamics code.
            For the sake of rapidity: envelope simulations only, hence no multiparticle or space-charge effects \alert{(ii) are these approximations too bold?}
         \end{block}

         \begin{block}{Presentation of the study case}
            \begin{figure}[hbtp]
               \includegraphics[width=\columnwidth,height=.2\columnwidth]{example-image-a}
               \caption{%
                  Structure of the JAEA--ADS.
                  \alert{We study the \nth{2} half of EllipR2: \( 940 \rightarrow \SI{1.5}{\GeV} \)}.
               }
               \label{fig:jaea_structure}
            \end{figure}
            JAEA--ADS is a project of ADS driven by a \SI{20}{\mA} \SI{1.5}{\GeV} proton beam.
            Cavities accelerating field \( \Eacc \) can be increased by \SI{20}{\percent} for compensation purposes.
            It's study is particularly interesting for us:
            \begin{enumerate}[(i)]
               \item a previous compensation study was realized using TraceWin and will serve as a benchmark;
               \item it's beam current is high, hence space-charge effects should be significant.
            \end{enumerate}
            We focus on the last \num{8} periods of the linac.
         \end{block}

         \begin{block}{Compensation with LightWin (+ TraceWin)}
            \begin{figure}[hbtp]
               \includegraphics[width=\columnwidth,height=.2\columnwidth]{example-image-a}
               \caption{Flowchart of the compensation process.}
               \label{fig:flowchart}
            \end{figure}
         \end{block}

         \begin{block}{LightWin (+ TraceWin) \emph{vs} TraceWin only from Ref.~: similar performances}
            \begin{table}
               \caption{%
                  Beam optics performance for this study.
                  TraceWin results are taken from Ref.~.
                  We outlined the beam optics that were better with TraceWin than with LightWin.
                  \alert{Both tools are equally performant\ldots~but LightWin is automatic.}
               }
               \label{tab:optics}
               \begin{center}
                  \begin{tabular}{@{} l c c c c c c c c @{}}
                     \toprule
                     & \multicolumn{4}{c}{$\Delta\varepsilon/\varepsilon_0$ [\si{\percent}]}     & \multicolumn{4}{c}{M} \\
                     \cmidrule(l){2-5}         \cmidrule(r){6-9}         
                     & \multicolumn{2}{c}{Transverse} &\multicolumn{2}{c}{Longitudinal} &\multicolumn{2}{c}{Transverse} &\multicolumn{2}{c}{Longitudinal} \\
                     \cmidrule(l){2-3}\cmidrule(rl){4-5}\cmidrule(rl){6-7}\cmidrule(r){8-9}
                     Faulty cavity & LightW  & TraceW       & LightW    & TraceW    & LightW & TraceW         & LightW  & TraceW                    \\
                     \midrule                                                                                                                                   
                     257  & \num{ 0.42}         &\num{1.07}        & \num{-1.11} &  \num{1.52} & \alert{\num{0.03}} & \alert{\num{0.02}} & \alert{\num{0.10}}& \alert{\num{0.09}}  \\
                     258  & \num{ 0.11}         &\num{0.88}        & \num{-0.27} &  \num{1.02} & \num{0.01} & \num{0.01}         & \num{0.02}& \num{0.09}                   \\
                     289  & \num{-0.01}         &\num{0.09}        & \num{-0.10} &  \num{0.13} & \num{0.00} & \num{0.01}         & \num{0.00}& \num{0.04}                   \\
                     290  & \num{-0.06}         &\num{-0.17}       & \num{ 0.10} &  \num{0.25} & \num{0.00} & \num{0.03}         & \num{0.01}& \num{0.08}                   \\
                     291  & \alert{\num{ 0.12}} &\alert{\num{0.04}}& \num{-0.03} &  \num{0.13} & \num{0.01} & \num{0.03}         & \num{0.03}& \num{0.09}                   \\
                     292  & \alert{\num{-0.09}} &\alert{\num{0.04}}& \num{ 0.10} &  \num{0.25} & \num{0.01} & \num{0.03}         & \num{0.02}& \num{0.09}                   \\
                     293  & \num{-0.02}         &\num{0.21}        & \num{ 0.17} &  \num{0.25} & \num{0.01} & \num{0.03}         & \num{0.02}& \num{0.09}                   \\
                     289--293& \num{ 0.18}         &\num{1.78}  & \num{-0.29} &  \num{-1.27} & \alert{\num{0.13}} & \alert{\num{0.02}}         & \alert{\num{0.35}}& \alert{\num{0.06}}                   \\
         % 289--293      & \num{-0.40}         &\num{1.78}        & \alert{\num{10.78}}  &  \alert{\num{-1.27}} & \alert{\num{0.08}} & \alert{\num{0.02}}         & \alert{\num{0.11}}& \alert{\num{0.06}}                   \\
                     \bottomrule
                  \end{tabular}
               \end{center}
            \end{table}
         \end{block}

      \end{column}
      \separatorcolumn
      \begin{column}{\colwidth}
         \begin{block}{Cavity error \#289: new cavity settings}
            Example of cavity error \#289, near the end of the linac.
            \emph{k-out-of-n} compensation method: \num{5} compensating cavities per error, as in \reffig{compensation_scheme}.
            \begin{figure}
               \centering
               \includegraphics[width=\columnwidth,height=.2\columnwidth]{example-image-a}
               \caption{%
                  Accelerating fields and synchronous phases of the cavities for the baseline design, the design from this study and the design from previous study Ref.
               }
            \end{figure}
         \end{block}

         \begin{block}{Cavity error \#289: densities}
            \begin{figure}
               \centering
               \includegraphics[width=\columnwidth,height=.2\columnwidth]{example-image-a}
               \includegraphics[width=\columnwidth,height=.2\columnwidth]{example-image-a}
               \includegraphics[width=\columnwidth,height=.2\columnwidth]{example-image-a}
               \includegraphics[width=\columnwidth,height=.2\columnwidth]{example-image-a}
               \caption{%
                  Beam densities in the transverse and longitudinal planes.
                  Calculated with TraceWin, \( 10^6 \) particles.
               }
            \end{figure}

            \begin{figure}[hbtp]
               \subfloat[][Input]{            \includegraphics[width=.24\columnwidth]{example-image-a}}
               \subfloat[][Output (baseline)]{ \includegraphics[width=.24\columnwidth]{example-image-a}}
               \subfloat[][Output (LightWin (+ TraceWin))]{\includegraphics[width=.24\columnwidth]{example-image-a}}
               \subfloat[][Output (TraceWin only Ref.)]{\includegraphics[width=.24\columnwidth]{example-image-a}}
               \caption{%
                  Yellow ellipses are \SI{100}{\percent} emittances.
               }
               \label{fig:emittance}
            \end{figure}
         \end{block}

            \begin{block}{Conclusions and perspectives}
               We showed that:
               \begin{itemize}
                  \item LightWin finds acceptable compensation settings;
                     \begin{itemize}
                        \item[\( \rightarrow \)] similar to TraceWin, but LightWin is automatic;
                     \end{itemize}
                  \item neglecting space-charge effects is acceptable in this section.
               \end{itemize}
               Future work:
               \begin{itemize}
                  \item perform similar study at the start of JAEA--ADS linac;
                     \begin{itemize}
                        \item lower energy \( \rightarrow \) space-charge effects more prominent;
                        \item if necessary: implement transverse dynamics, space-charge in LightWin;
                     \end{itemize}
                  \item new optimisation algorithms;
                  \item explore new compensation paradigm (\eg for SPIRAL2):
                     \begin{itemize}
                        \item no increase in accelerating field\ldots
                        \item \ldots~but use more cavities for compensation: \alert{global compensation}.
                     \end{itemize}
               \end{itemize}
               % \todo{Shift items to right in enumerate}
               % \todo{reduce space above jaea fig}
            \end{block}

            % \printbibliography
         \end{column}
         \separatorcolumn
      \end{columns}
\end{frame}

\end{document}
